\documentclass{../../ktane-mod}


\begin{document}

\begin{module}{
  moduleName=Memory,
  indexString=Memory,
  imageResource=memory_img.pdf,
  interactions=\keysymbol
}
{
  Memory is a fragile thing but so is everything else when a bomb goes off, so pay attention!
}
  \begin{bulletlist}
    \bulletitem{Press the correct button to progress the module to the next stage.
    Complete all stages A-E to disarm the module.}
    \bulletitem{Pressing an incorrect button will reset the module back to stage A and record a strike.}
    \bulletitem{Button positions are ordered from left to right.}
  \end{bulletlist}

  Use the table to determine which button to press in which stage.
  As in the last two columns below, take note of the label and the position of the pressed button.
  If the instruction references a position, press the button in that position.
  If the instruction references a label, press the button with that label.
  The colors serve as a quick reference and do not appear in the module.

  \renewcommand{\arraystretch}{1.7}
  \begin{NiceTabular}{
    >{\centering\arraybackslash}m{2.7cm}
    >{\centering\arraybackslash}m{1.7cm}
    >{}m{3.7cm}
    >{\centering\arraybackslash}m{2.7cm}
    >{\centering\arraybackslash}m{2.7cm}
  }
    \CodeBefore
      % Color coding for Label column (stages)
      \cellcolor{red}{2-4,3-4,4-4,5-4}            % Stage A label blocks - red
      \cellcolor{green}{6-4,7-4,8-4,9-4}          % Stage B label blocks - green
      \cellcolor{blue}{10-4,11-4,12-4,13-4}       % Stage C label blocks - blue
      \cellcolor{yellow}{14-4,15-4,16-4,17-4}     % Stage D label blocks - yellow
      % Color coding for "Label from X" cells in instruction column
      \cellcolor{red}{11-3,18-3}                  % "Label from A" - red
      \cellcolor{green}{10-3,19-3}                % "Label from B" - green
      \cellcolor{blue}{21-3}                      % "Label from C" - blue
      \cellcolor{yellow}{20-3}                    % "Label from D" - yellow
      % Color coding for Position column (stages A and B)
      \cellcolor{gray}{2-5,3-5,4-5,5-5}     % Stage A position blocks - lightgray
      \cellcolor{lightpurple}{6-5,7-5,8-5,9-5}   % Stage B position blocks - lightpurple
      % Color coding for "Position from X" cells in instruction column
      \cellcolor{gray}{7-3,9-3,14-3}        % "Position from A" - lightgray
      \cellcolor{lightpurple}{16-3,17-3}          % "Position from B" - lightpurple
    \Body
    Stage                 & Display & \centering Instruction          & Label         & Position \\
  %  Stage A
    \Block{4-1}{\LARGE A} & 1       & \Block[l]{2-1}{Position 2}      & \Block{4-1}{} & \Block{4-1}{}\\
                          & 2       &                                 &               & \\
                          & 3       & Position 3                      &               & \\
                          & 4       & Position 4                      &               & \\
  % Stage B
    \Block{4-1}{\LARGE B} & 1       & Label "4"                       & \Block{4-1}{} & \Block{4-1}{}\\
                          & 2       & Position from A                 &               & \\
                          & 3       & Position 1                      &               & \\
                          & 4       & Position from A                 &               & \\
  % Stage C
    \Block{4-1}{\LARGE C} & 1       & Label from B                    & \Block{4-1}{} & \\
                          & 2       & Label from A                    &               & \\
                          & 3       & Position 3                      &               & \\
                          & 4       & Label "4"                       &               & \\
  % Stage D
    \Block{4-1}{\LARGE D} & 1       & Position from A                 & \Block{4-1}{} & \\
                          & 2       & Position 1                      &               & \\
                          & 3       & \Block[l]{2-1}{Position from B} &               & \\
                          & 4       &                                 &               & \\
  % Stage E
    \Block{4-1}{\LARGE E} & 1       & Label from A                    &               & \\
                          & 2       & Label from B                    &               & \\
                          & 3       & Label from D                    &               & \\
                          & 4       & Label from C                    &               & \\
  \CodeAfter
    % horizontal lines
    \tikz \draw[line width=3pt, line cap=rect] (1-|1) -- (1-|last);
    \tikz \draw[line width=3pt, line cap=rect] (2-|1) -- (2-|last);
    \tikz \draw[line width=1pt] (3-|2) -- (3-|3);
    \tikz \draw[line width=1pt] (4-|2) -- (4-|4);
    \tikz \draw[line width=1pt] (5-|2) -- (5-|4);

    \tikz \draw[line width=3pt, line cap=rect] (6-|1) -- (6-|last);
    \tikz \draw[line width=1pt] (7-|2) -- (7-|4);
    \tikz \draw[line width=1pt] (8-|2) -- (8-|4);
    \tikz \draw[line width=1pt] (9-|2) -- (9-|4);

    \tikz \draw[line width=3pt, line cap=rect] (10-|1) -- (10-|last);
    \tikz \draw[line width=1pt] (11-|2) -- (11-|4);
    \tikz \draw[line width=1pt] (12-|2) -- (12-|4);
    \tikz \draw[line width=1pt] (13-|2) -- (13-|4);

    \tikz \draw[line width=3pt, line cap=rect] (14-|1) -- (14-|5);
    \tikz \draw[line width=1pt] (15-|2) -- (15-|4);
    \tikz \draw[line width=1pt] (16-|2) -- (16-|4);
    \tikz \draw[line width=1pt] (17-|2) -- (17-|3);

    \tikz \draw[line width=3pt, line cap=rect] (18-|1) -- (18-|5);
    \tikz \draw[line width=1pt] (19-|2) -- (19-|4);
    \tikz \draw[line width=1pt] (20-|2) -- (20-|4);
    \tikz \draw[line width=1pt] (21-|2) -- (21-|4);

    \tikz \draw[line width=3pt, line cap=rect] (22-|1) -- (22-|4);
    % vertical lines
    \tikz \draw[line width=3pt, line cap=rect] (1-|1) -- (last-|1);
    \tikz \draw[line width=1pt] (1-|2) -- (last-|2);
    \tikz \draw[line width=1pt] (1-|3) -- (last-|3);
    \tikz \draw[line width=1pt] (1-|4) -- (18-|4);
    \tikz \draw[line width=3pt, line cap=rect] (18-|4) -- (last-|4);
    \tikz \draw[line width=1pt] (1-|5) -- (10-|5);
    \tikz \draw[line width=3pt, line cap=rect] (10-|5) -- (18-|5);
    \tikz \draw[line width=3pt, line cap=rect] (1-|6) -- (10-|6);
  \end{NiceTabular}
  \renewcommand{\arraystretch}{1.0}

\end{module}

\end{document}
