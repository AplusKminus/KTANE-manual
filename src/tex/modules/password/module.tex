\documentclass{../../ktane-mod}

\setlength{\parindent}{0pt}
\setlength{\parskip}{10pt}

\begin{document}

\begin{module}{
  moduleName=Passwords,
  indexString=Passwords,
  imageResource=password_img.pdf,
  interactions=\keysymbol
}
{
  Fortunately this password doesn't seem to meet standard government security requirements: 22 characters, mixed case, numbers in random order without any palindromes above length 3.
}
  \begin{bulletlist}
    \bulletitem{The buttons above and below each letter will cycle through the possibilities for that position.}
    \bulletitem{Only one combination of the available letters will match a password from the list.}
    \bulletitem{Press the submit button once the correct word has been set.}
  \end{bulletlist}

  \begin{wraptable}{r}{6cm}
    \renewcommand{\arraystretch}{1.25}
    \begin{NiceTabular}{
    >{\centering\arraybackslash}m{0.4cm}
    >{\centering\arraybackslash}m{0.4cm}
    >{\centering\arraybackslash}m{0.4cm}
    >{\centering\arraybackslash}m{2cm}
    }[hvlines]
      \textbf{1.}    & \textbf{4.}    & \textbf{2.}    & \textbf{Password}\\
      \Block{3-1}{A} & E              & F              & after\\
                     & I              & G              & again\\
                     & U              & B              & about\\
      B              & O              & E              & below\\
      C              & L              & O              & could\\
      E              & R              & V              & every\\
      \Block{2-1}{F} & N              & O              & found\\
                     & S              & I              & first\\
      G              & A              & R              & great\\
      H              & S              & O              & house\\
      \Block{2-1}{L} & G              & A              & large\\
                     & R              & E              & learn\\
      N              & E              & E              & never\\
      O              & E              & T              & other\\
      \Block{3-1}{P} & C              & \Block{2-1}{L} & place\\
                     & \Block{2-1}{N} &                & plant\\
                     &                & O              & point\\
      R              & H              & I              & right\\
      \Block{5-1}{S} & D              & T              & study\\
                     & \Block{3-1}{L} & M              & small\\
                     &                & P              & spell\\
                     &                & T              & still\\
                     & N              & O              & sound\\
      \Block{6-1}{T} & E              & \Block{6-1}{H} & three\\
                     & I              &                & their\\
                     & \Block{2-1}{N} &                & thing\\
                     &                &                & think\\
                     & R              &                & there\\
                     & S              &                & these\\
      \Block{6-1}{W} & C              & H              & which\\
                     & E              & A              & water\\
                     & \Block{2-1}{L} & \Block{2-1}{O} & world\\
                     &                &                & would\\
                     & R              & H              & where\\
                     & T              & R              & write\\
    \CodeAfter
      % Bold outer border (2pt lines around entire table)
      \tikz \draw[line width=2pt, line cap=rect] (1-|1) -- (1-|last);   % Top
      \tikz \draw[line width=2pt, line cap=rect] (last-|1) -- (last-|last); % Bottom
      \tikz \draw[line width=2pt, line cap=rect] (1-|1) -- (last-|1);   % Left
      \tikz \draw[line width=2pt, line cap=rect] (1-|last) -- (last-|last); % Right
      
      % Bold lines before each new letter in first column (2pt)
      \tikz \draw[line width=2pt, line cap=rect] (2-|1) -- (2-|last);   % Before A
      \tikz \draw[line width=2pt, line cap=rect] (5-|1) -- (5-|last);   % Before B
      \tikz \draw[line width=2pt, line cap=rect] (6-|1) -- (6-|last);   % Before C
      \tikz \draw[line width=2pt, line cap=rect] (7-|1) -- (7-|last);   % Before E
      \tikz \draw[line width=2pt, line cap=rect] (8-|1) -- (8-|last);   % Before F
      \tikz \draw[line width=2pt, line cap=rect] (10-|1) -- (10-|last); % Before G
      \tikz \draw[line width=2pt, line cap=rect] (11-|1) -- (11-|last); % Before H
      \tikz \draw[line width=2pt, line cap=rect] (12-|1) -- (12-|last); % Before L
      \tikz \draw[line width=2pt, line cap=rect] (14-|1) -- (14-|last); % Before N
      \tikz \draw[line width=2pt, line cap=rect] (15-|1) -- (15-|last); % Before O
      \tikz \draw[line width=2pt, line cap=rect] (16-|1) -- (16-|last); % Before P
      \tikz \draw[line width=2pt, line cap=rect] (19-|1) -- (19-|last); % Before R
      \tikz \draw[line width=2pt, line cap=rect] (20-|1) -- (20-|last); % Before S
      \tikz \draw[line width=2pt, line cap=rect] (25-|1) -- (25-|last); % Before T
      \tikz \draw[line width=2pt, line cap=rect] (31-|1) -- (31-|last); % Before W
    \end{NiceTabular}
  \end{wraptable}
\textheading{Step 1}\medskip\\
  Ask the defuser to spell out the letters of the \YELLOW[first] and \YELLOW[fourth] position.
  Write them down as in the example below:

  \quad\begin{tabular}{ll}
    \textbf{1.} & \textbf{4.}\\
    \hline
    B  & E \\
    X  & S \\
    A  & Z \\
    L  & Q \\
    U  & V \\
    W  & L \\
  \end{tabular}

  \textheading{Step 2}\medskip\\
  Look for potential matches in the table on the right.

  \textit{For the above examples:}

  \quad AE --- after\\
  \quad WE --- water\\
  \quad WL --- world\\
  \quad WL --- would

  \textheading{Step 3}\medskip\\
  If too many matches are found and the defuser does not have enough time to test them, ask them to spell out the second position as well.

  There are only four words that need even further differentiation: thing, think, world and would.\bigskip\\

  \textbf{Hint:} It is helpful if the defuser does not report the letters J, K, Q, X, Y or Z to the expert, as they do not appear in the positions 1, 4 or 2 of any password.
\end{module}
\end{document}
