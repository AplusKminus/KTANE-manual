\begin{module}{
  moduleName=Complicated Wires,
  indexString=Complicated Wires,
  imageResource=complicated_wires.png
}
{
  These wires aren't like the others. Some have stripes! That makes them completely different. The good news is that we've found a concise set of instructions on what to do about it! Maybe too concise...
}

% Table parameters
\newlength{\cellwidth}
\newlength{\cellheight}
\newlength{\cellheightwithoutpadding}
\setlength{\cellwidth}{1cm}
\setlength{\cellheight}{1.13cm}
\setlength{\cellheightwithoutpadding}{1cm}
\newlength{\lastcellwidth}
\setlength{\lastcellwidth}{2.1cm}
% Calculated dimensions
\newlength{\fivecellwidth}
\setlength{\fivecellwidth}{5.69cm}
\newlength{\twocellheight}
\setlength{\twocellheight}{2\cellheight}
% Symbol positioning offsets
\newlength{\circleoffset}
\setlength{\circleoffset}{0.20\cellwidth}
\newlength{\staroffset}
\setlength{\staroffset}{0.05\cellwidth}
% Line thickness
\newlength{\tablelinewidth}
\setlength{\tablelinewidth}{1pt}

\begin{itemize}
  \item[$\bullet$] Check the 4 sections one after another.
  \item[$\bullet$] If a section applies, cut all wires matching any of its conditions.
\end{itemize}

Hints:
\begin{itemize}
  \item[$\bullet$] \PURPLE means that the wire has both \RED and \BLUE in it.
  \item[$\bullet$] Plain \WHITE means that the wire has neither \RED nor \BLUE in it.
  \item[$\bullet$] \RED means that the wire has no \BLUE in it.
  \item[$\bullet$] \BLUE means that the wire has no \RED in it.
\end{itemize}

\vspace{2cm}

\begin{minipage}[t]{0.48\textwidth}
% Left Column, Top
\textheading{Always:}
\begin{enumerate}
  \renewcommand{\labelenumi}{\alph{enumi}.}
  \item LED off, Plain \WHITE
  \item LED off, \RED with STAR
\end{enumerate}

\vspace{0.5cm}

% Left Column, Bottom
\textheading{If there are 2+ Batteries:}
\begin{enumerate}
  \renewcommand{\labelenumi}{\alph{enumi}.}
  \item LED on, Plain \WHITE with STAR
  \item LED on, \RED
\end{enumerate}
\end{minipage}
\hfill
\begin{minipage}[t]{0.48\textwidth}
% Right Column, Top
\textheading{If there is a Parallel Port:}
\begin{enumerate}
  \renewcommand{\labelenumi}{\alph{enumi}.}
  \item LED on, \BLUE
  \item LED off, \PURPLE with STAR
\end{enumerate}

\vspace{0.5cm}

% Right Column, Bottom
\textheading{If the Last Digit of the Serial\# is Even:}
\begin{enumerate}
  \renewcommand{\labelenumi}{\alph{enumi}.}
  \setcounter{enumi}{2}
  \item LED off, no STAR
  \item \PURPLE, no STAR
\end{enumerate}
\end{minipage}

\vspace{1.5cm}

\begin{center}
\large\textheading{Compact KV-Diagram}
\end{center}

\newcommand{\rowheight}{\rule{0pt}{\cellheightwithoutpadding}} % Adjust height here
\begin{NiceTabular}{
    >{\centering\arraybackslash}m{\lastcellwidth}
    >{\centering\arraybackslash}m{\cellwidth}
    >{\centering\arraybackslash}m{\cellwidth}
    >{\centering\arraybackslash}m{\cellwidth}
    >{\centering\arraybackslash}m{\cellwidth}
    >{\centering\arraybackslash}m{\cellwidth}
    >{\centering\arraybackslash}m{\cellwidth}
    >{\centering\arraybackslash}m{\cellwidth}
    >{\centering\arraybackslash}m{\cellwidth}
    >{\centering\arraybackslash}m{\lastcellwidth}
}
  \CodeBefore
    % Red stripe pattern from 2-5 to 5-9 (with white background)
    \tikz \fill[blue!70] (2-|6) rectangle ++(\fivecellwidth,-\cellheight);
    \tikz \fill[pattern=diagonal stripes, pattern color=red!70] (2-|6) rectangle ++(\fivecellwidth,-\cellheight);
    \tikz \fill[pattern=diagonal stripes, pattern color=red!70] (3-|6) rectangle ++(\fivecellwidth,-\twocellheight);
    \tikz \fill[blue!70] (5-|6) rectangle ++(\fivecellwidth,-\cellheight);
    \tikz \fill[pattern=diagonal stripes, pattern color=red!70] (5-|6) rectangle ++(\fivecellwidth,-\cellheight);
    % Blue coloring using blocks (excluding cells with red stripes)
    \cellcolor{blue!70}{2-3,2-5,5-3,5-5}
    \cellcolor{red!70}{3-7,3-9,4-7,4-9}
    % Diagonal stripe pattern for cell 2-2 (with white background)
    \tikz \fill[pattern=diagonal stripes, pattern color=blue!70] (2-|2) rectangle ++(\fivecellwidth,-\cellheight);
    \tikz \fill[pattern=diagonal stripes, pattern color=blue!70] (5-|2) rectangle ++(\fivecellwidth,-\cellheight);
  \Body
  % Row 1 -- 1 empty cell, 2 merged cells, 2 merged cells, 1 empty cell
  \rowheight & \Block{1-4}{No RED} & & & & \Block{1-4}{Has RED} & & & & \\
  % Row 2 -- 1 2-row cell with below, 2 merged cells, 3 individual cells
  \rowheight\Block{2-1}{LED on} & \Block{1-4}{Parallel Port} & & & & \Block{1-2}{Don't\\ Cut} & & \Block{1-2}{Serial\#\\ even} & & Has BLUE \\
  % Row 3 -- 1 2-row cell from above, 1 individual cell, 3 merged cells, 1 2-row cell with below
  \rowheight & \Block{1-2}{Don't\\ Cut} & & \Block{1-6}{Two or more Batteries} & & & & & & \Block{2-1}{No BLUE} \\
  % Row 4 -- 1 2-row cell with below, 3 merged cells, 1 2-row cell with below, 1 2-row cell with above
  \rowheight\Block{2-1}{LED off} & \Block{1-6}{Cut} & & & & & & \Block{2-2}{Serial\#\\ even} & & \\
  % Row 5 -- 1 2-row cell from above, 3 individual cells, 1 2-row cell with above, 1 individual cell
  \rowheight & \Block{1-2}{Serial\#\\ even} & & \Block{1-2}{Don't\\ Cut} & & \Block{1-2}{Parallel\\ Port} & & & & Has BLUE \\
  % Row 6 -- 2 individual cells, 2 merged cells, 2 individual cells
  \rowheight & \Block{1-2}{No STAR} & & \Block{1-4}{STAR} & & & & \Block{1-2}{No STAR} & & \\
  \CodeAfter
  % Horizontal lines (inside only, accounting for merged cells)
  \tikz \draw[line width=\tablelinewidth] (2-|1) -- (2-|last);
  \tikz \draw[line width=\tablelinewidth] (3-|2) -- (3-|last);
  \tikz \draw[line width=\tablelinewidth] (4-|1) -- (4-|10);
  \tikz \draw[line width=\tablelinewidth] (5-|2) -- (5-|8);
  \tikz \draw[line width=\tablelinewidth] (5-|10) -- (5-|last);
  \tikz \draw[line width=\tablelinewidth] (6-|1) -- (6-|last);
  % Vertical lines (inside only, accounting for merged cells)
  \tikz \draw[line width=\tablelinewidth] (1-|2) -- (last-|2);
  \tikz \draw[line width=\tablelinewidth] (3-|4) -- (4-|4);
  \tikz \draw[line width=\tablelinewidth] (5-|4) -- (last-|4);
  \tikz \draw[line width=\tablelinewidth] (1-|6) -- (3-|6);
  \tikz \draw[line width=\tablelinewidth] (5-|6) -- (6-|6);
  \tikz \draw[line width=\tablelinewidth] (2-|8) -- (3-|8);
  \tikz \draw[line width=\tablelinewidth] (4-|8) -- (last-|8);
  \tikz \draw[line width=\tablelinewidth] (1-|10) -- (last-|10);
  % Small yellow circles in top left of cells row 2-3/col 2-5 (mapped to new column structure)
  \tikz \draw[fill=yellow!30, draw=black, line width=0.5pt] (2-|2) ++(\circleoffset,-\circleoffset) circle (4pt);
  \tikz \draw[fill=yellow!30, draw=black, line width=0.5pt] (2-|4) ++(\circleoffset,-\circleoffset) circle (4pt);
  \tikz \draw[fill=yellow!30, draw=black, line width=0.5pt] (2-|6) ++(\circleoffset,-\circleoffset) circle (4pt);
  \tikz \draw[fill=yellow!30, draw=black, line width=0.5pt] (2-|8) ++(\circleoffset,-\circleoffset) circle (4pt);
  \tikz \draw[fill=yellow!30, draw=black, line width=0.5pt] (3-|2) ++(\circleoffset,-\circleoffset) circle (4pt);
  \tikz \draw[fill=yellow!30, draw=black, line width=0.5pt] (3-|4) ++(\circleoffset,-\circleoffset) circle (4pt);
  \tikz \draw[fill=yellow!30, draw=black, line width=0.5pt] (3-|6) ++(\circleoffset,-\circleoffset) circle (4pt);
  \tikz \draw[fill=yellow!30, draw=black, line width=0.5pt] (3-|8) ++(\circleoffset,-\circleoffset) circle (4pt);
  % Black 5-pointed stars in bottom right of cells row 2-5/col 3-4 (moved 2 columns right)
  \tikz \node at (3-|6) [anchor=south east, xshift=-\staroffset, yshift=\staroffset] {\tikz \node[star, star points=5, star point ratio=2.5, inner sep=0pt, minimum size=8pt, fill=black] {};};
  \tikz \node at (3-|8) [anchor=south east, xshift=-\staroffset, yshift=\staroffset] {\tikz \node[star, star points=5, star point ratio=2.5, inner sep=0pt, minimum size=8pt, fill=black] {};};
  \tikz \node at (4-|6) [anchor=south east, xshift=-\staroffset, yshift=\staroffset] {\tikz \node[star, star points=5, star point ratio=2.5, inner sep=0pt, minimum size=8pt, fill=black] {};};
  \tikz \node at (4-|8) [anchor=south east, xshift=-\staroffset, yshift=\staroffset] {\tikz \node[star, star points=5, star point ratio=2.5, inner sep=0pt, minimum size=8pt, fill=black] {};};
  \tikz \node at (5-|6) [anchor=south east, xshift=-\staroffset, yshift=\staroffset] {\tikz \node[star, star points=5, star point ratio=2.5, inner sep=0pt, minimum size=8pt, fill=black] {};};
  \tikz \node at (5-|8) [anchor=south east, xshift=-\staroffset, yshift=\staroffset] {\tikz \node[star, star points=5, star point ratio=2.5, inner sep=0pt, minimum size=8pt, fill=black] {};};
  \tikz \node at (6-|6) [anchor=south east, xshift=-\staroffset, yshift=\staroffset] {\tikz \node[star, star points=5, star point ratio=2.5, inner sep=0pt, minimum size=8pt, fill=black] {};};
  \tikz \node at (6-|8) [anchor=south east, xshift=-\staroffset, yshift=\staroffset] {\tikz \node[star, star points=5, star point ratio=2.5, inner sep=0pt, minimum size=8pt, fill=black] {};};
\end{NiceTabular}

\end{module}
