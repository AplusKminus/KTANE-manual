\begin{module}{
  moduleName=The Button,
  indexString=Button,
  imageResource=button.png
}
{
  You might think that a button telling you to press it is pretty straightforward.
  That’s the kind of thinking that gets people exploded.

  See the section on edgework for indicator and battery identification reference.
}
  Use the edgework to identify which of the tables below should be used to disarm The Button.
  Use the first table whose condition in its top left cell matches the widgets present in the edgework.

  %% 0-1 Battery
  %
  %
  \begin{NiceTabular}{|
      >{\centering\arraybackslash}p{4cm} |
      >{\centering\arraybackslash}p{1.75cm}
      >{\centering\arraybackslash}p{1.75cm}
      >{\centering\arraybackslash}p{3.5cm}
      >{\centering\arraybackslash}p{3.5cm} |}
    \hline
    % Table head
    \cellcolor{lightgray}\marginbox{0.1cm}{\large0-1 Battery} &
    \cellcolor{blue}BLUE &
    \cellcolor{yellow}YELLOW &
    \cellcolor{red}RED &
    \cellcolor{lightgray}WHITE \\
    \hline
    \cellcolor{lightgray}"Detonate" & & & \multirow{3}*{HOLD} \\
    \cellcolor{lightgray}"Abort" & \multicolumn{4}{c}{} \\
    \cellcolor{lightgray}"Press" & \multicolumn{4}{c}{} \\
    \cellcolor{lightgray}"Hold" & \multicolumn{2}{c}{} & PRESS \& RELEASE & \\
    \hline
    \CodeAfter
    \tikz \draw (last-|4) -| (5-|4) -| (5-|5) -| (last-|5) ;
    %top seperator lines
    \tikz \draw (1-|3) |- (2-|3) ;
    \tikz \draw (1-|4) |- (2-|4) ;
    \tikz \draw (1-|5) |- (2-|5) ;
    %left seperator lines
    \tikz \draw (3-|1) |- (3-|2) ;
    \tikz \draw (4-|1) |- (4-|2) ;
    \tikz \draw (5-|1) |- (5-|2) ;
  \end{NiceTabular}

  %% lit FRK and 3+ Batt
  %
  %
  \begin{NiceTabular}{|
      >{\centering\arraybackslash}p{4cm} |
      >{\centering\arraybackslash}p{1.75cm}
      >{\centering\arraybackslash}p{1.75cm}
      >{\centering\arraybackslash}p{3.5cm}
      >{\centering\arraybackslash}p{3.5cm} |}
    \hline
    % Table head
    \cellcolor{lightgray}\marginbox{0.1cm}{\large lit FRK and 3+ Batt} &
    \cellcolor{blue}BLUE &
    \cellcolor{yellow}YELLOW &
    \cellcolor{red}RED &
    \cellcolor{lightgray}WHITE \\
    \hline
    \cellcolor{lightgray}"Detonate" &   & & \multirow{4}*{PRESS \& RELEASE} & \\
    \cellcolor{lightgray}"Abort" & HOLD & & & \multirow{3}*{\parbox{3.5cm}{\centering lit CAR - HOLD otherwise PRESS \& RELEASE}} \\
    \cellcolor{lightgray}"Press" &      & &                                 & \\
    \cellcolor{lightgray}"Hold"  &      & &                                 & \\
    \hline
    \CodeAfter
    %top seperator lines
    \tikz \draw (1-|3) |- (2-|3) ;
    \tikz \draw (1-|4) |- (2-|4) ;
    \tikz \draw (1-|5) |- (2-|5) ;
    %left seperator lines
    \tikz \draw (3-|1) |- (3-|3) |- (4-|3) |- (4-|1) ;
    \tikz \draw (5-|1) |- (5-|2) ;
    \tikz \draw (last-|5) |- (3-|5) |- (3-|last) ;
  \end{NiceTabular}

  %% Otherwise
  %
  %
  \begin{NiceTabular}{|
      >{\centering\arraybackslash}p{4cm} |
      >{\centering\arraybackslash}p{1.75cm}
      >{\centering\arraybackslash}p{1.75cm}
      >{\centering\arraybackslash}p{3.5cm}
      >{\centering\arraybackslash}p{3.5cm} |}
    \hline
    % Table head
    \cellcolor{lightgray}\marginbox{0.1cm}{\large Otherwise} &
    \cellcolor{blue}BLUE &
    \cellcolor{yellow}YELLOW &
    \cellcolor{red}RED &
    \cellcolor{lightgray}WHITE \\
    \hline
    \cellcolor{lightgray}"Detonate" & \multicolumn{4}{c}{PRESS \& RELEASE} \\
    \cellcolor{lightgray}"Abort" & & & \multirow{2}*{HOLD} & \\
    \cellcolor{lightgray}"Press" & \multicolumn{4}{c}{} \\
    \cellcolor{lightgray}"Hold" & \multicolumn{2}{c}{} & PRESS \& RELEASE & \\
    \hline
    \CodeAfter
    \tikz \draw (last-|4) -| (5-|4) -| (5-|5) -| (last-|5) ;
    %top seperator lines
    \tikz \draw (1-|3) |- (2-|3) ;
    \tikz \draw (1-|4) |- (2-|4) ;
    \tikz \draw (1-|5) |- (2-|5) ;
    %left seperator lines
    \tikz \draw (3-|1) |- (3-|last) ;
    \tikz \draw (4-|1) |- (4-|2) ;
    \tikz \draw (5-|1) |- (5-|2) ;
  \end{NiceTabular}

  \subsection*{Releasing a Held Button:}
  If you start holding the button down, a colored strip will light up on the
  right side of the module. Based on its color you must release the button at
  a specific point in time:

    %% Otherwise
  %
  %
  \begin{NiceTabular}{|
      >{\centering\arraybackslash}p{4cm} |
      >{\centering\arraybackslash}p{3.5cm} |
      >{\centering\arraybackslash}p{3.5cm} |
      >{\centering\arraybackslash}p{3.5cm} |}
    \hline
    % Table head
    \cellcolor{lightgray}\marginbox{0.1cm}{\large HOLDING} &
    \cellcolor{blue}BLUE &
    \cellcolor{yellow}YELLOW &
    Otherwise \\
    \hline
    \cellcolor{lightgray}"Release when X in timer" &
    \vspace{0cm}4 &
    \vspace{0cm}5 &
    \vspace{0cm}1 \\
    \hline
  \end{NiceTabular}

\end{module}
